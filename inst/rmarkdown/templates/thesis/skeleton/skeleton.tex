\documentclass[]{dissertateCICESE}
\usepackage{lmodern}
\usepackage{amssymb,amsmath}
\usepackage{ifxetex,ifluatex}
\usepackage{fixltx2e} % provides \textsubscript
\ifnum 0\ifxetex 1\fi\ifluatex 1\fi=0 % if pdftex
  \usepackage[T1]{fontenc}
  \usepackage[utf8]{inputenc}
\else % if luatex or xelatex
  \ifxetex
    \usepackage{mathspec}
  \else
    \usepackage{fontspec}
  \fi
  \defaultfontfeatures{Ligatures=TeX,Scale=MatchLowercase}
    \setmainfont[]{OpenSans}
\fi
% use upquote if available, for straight quotes in verbatim environments
\IfFileExists{upquote.sty}{\usepackage{upquote}}{}
% use microtype if available
\IfFileExists{microtype.sty}{%
\usepackage{microtype}
\UseMicrotypeSet[protrusion]{basicmath} % disable protrusion for tt fonts
}{}
\usepackage[top=0.8in,bottom=0.8in,right=0.8in,left=1.18in]{geometry}
\usepackage{hyperref}
\hypersetup{unicode=true,
            pdftitle={Predicción de sitios de acoplamiento de factores de transcripción a partir del perfil de activación de la histona H3K27ac},
            pdfauthor={César Miguel Valdez Córdova},
            pdfborder={0 0 0},
            breaklinks=true}
\urlstyle{same}  % don't use monospace font for urls
\usepackage{graphicx,grffile}
\makeatletter
\def\maxwidth{\ifdim\Gin@nat@width>\linewidth\linewidth\else\Gin@nat@width\fi}
\def\maxheight{\ifdim\Gin@nat@height>\textheight\textheight\else\Gin@nat@height\fi}
\makeatother
% Scale images if necessary, so that they will not overflow the page
% margins by default, and it is still possible to overwrite the defaults
% using explicit options in \includegraphics[width, height, ...]{}
\setkeys{Gin}{width=\maxwidth,height=\maxheight,keepaspectratio}
\IfFileExists{parskip.sty}{%
\usepackage{parskip}
}{% else
\setlength{\parindent}{0pt}
\setlength{\parskip}{6pt plus 2pt minus 1pt}
}
\setlength{\emergencystretch}{3em}  % prevent overfull lines
\providecommand{\tightlist}{%
  \setlength{\itemsep}{0pt}\setlength{\parskip}{0pt}}
\setcounter{secnumdepth}{0}
% Redefines (sub)paragraphs to behave more like sections
\ifx\paragraph\undefined\else
\let\oldparagraph\paragraph
\renewcommand{\paragraph}[1]{\oldparagraph{#1}\mbox{}}
\fi
\ifx\subparagraph\undefined\else
\let\oldsubparagraph\subparagraph
\renewcommand{\subparagraph}[1]{\oldsubparagraph{#1}\mbox{}}
\fi

%%% Use protect on footnotes to avoid problems with footnotes in titles
\let\rmarkdownfootnote\footnote%
\def\footnote{\protect\rmarkdownfootnote}

%%% Change title format to be more compact
\usepackage{titling}

% Create subtitle command for use in maketitle
\newcommand{\subtitle}[1]{
  \posttitle{
    \begin{center}\large#1\end{center}
    }
}

\setlength{\droptitle}{-2em}

  \title{Predicción de sitios de acoplamiento de factores de transcripción a
partir del perfil de activación de la histona H3K27ac}
    \pretitle{\vspace{\droptitle}\centering\huge}
  \posttitle{\par}
    \author{César Miguel Valdez Córdova}
    \preauthor{\centering\large\emph}
  \postauthor{\par}
      \predate{\centering\large\emph}
  \postdate{\par}
    \date{December 11, 2018}

\newcommand{\yeardegree}{ 2019 } \newcommand{\degree}{ Maestria en Ciencias }
 \newcommand{\degreetitle}{ Maestro en Ciencias }
 \newcommand{\field}{ Ciencias de la Computación }
 \newcommand{\advisor}{ Advisor }
 \newcommand{\coadvisor}{ Coadvisor }
 \newcommand{\committeeone}{ Committee Member 1 }
 \newcommand{\committeetwo}{ Committee Member 2 }
 \newcommand{\committeethree}{ Committee Member 3 }
 \newcommand{\coordinator}{ Program Coordinator }
 \newcommand{\director}{ director }
 % Tables
      \usepackage{booktabs}
      \usepackage{threeparttable}
      \usepackage{array}
      \newcolumntype{x}[1]{%
      >{\centering\arraybackslash}m{#1}}%
      \usepackage{placeins}
      \usepackage{chngcntr}
      \counterwithin{figure}{chapter}
      \counterwithin{table}{chapter}
      \usepackage[makeroom]{cancel}

\begin{document}
\maketitle

\pagenumbering{roman}
\pagestyle{empty}

\newpage
\pagestyle{fancy}
\fancyhead[L]{Abstract}
\fancyhead[R]{\thepage}
\fancyfoot[C]{}
\chapter*{ABSTRACT}
\addcontentsline{toc}{section}{Abstract}

\newpage
\fancyhead[L]{Dedication}
\fancyhead[R]{\thepage}
\fancyfoot[C]{}
\chapter*{DEDICATION}
\addcontentsline{toc}{section}{Dedication}

Dedicate it.

\newpage
\fancyhead[L]{Acknowledgments}
\fancyhead[R]{\thepage}
\fancyfoot[C]{}
\chapter*{ACKNOWLEDGEMENTS}
\addcontentsline{toc}{section}{Acknowledgments}

Acknowledge them.

\newpage
\fancyhead[L]{Table of Contents}
\fancyhead[R]{\thepage}
\fancyfoot[C]{}
\tableofcontents

\newpage
\fancyhead[L]{List of Tables}
\fancyhead[R]{\thepage}
\fancyfoot[C]{}
\listoftables

\newpage
\fancyhead[L]{List of Figures}
\fancyhead[R]{\thepage}
\fancyfoot[C]{}
\listoffigures

\newpage
\pagenumbering{arabic}

\newpage
\fancyhead[L]{Introduction}
\fancyhead[R]{\thepage}
\fancyfoot[C]{}

\chapter{INTRODUCTION}

Introduce the thing.

\FloatBarrier

\newpage
\fancyhead[L]{Chapter 2's Title}
\fancyhead[R]{\thepage}
\fancyfoot[C]{}

\chapter{Chapter 2's Title}

Tells us more.

\FloatBarrier

\newpage
\fancyhead[L]{Chapter 3's Title}
\fancyhead[R]{\thepage}
\fancyfoot[C]{}

\chapter{Chapter 3's Title}

Don't stop now.

\FloatBarrier
\newpage
\fancyhead[L]{Chapter 4's Title}
\fancyhead[R]{\thepage}
\fancyfoot[C]{}

\chapter{Chapter 4's Title}

Keep it going.

\FloatBarrier
\newpage
\fancyhead[L]{Chapter 5's Title}
\fancyhead[R]{\thepage}
\fancyfoot[C]{}

\chapter{Chapter 5's Title}

Well done.

\FloatBarrier

\newpage
\fancyhead[L]{References}
\fancyhead[R]{\thepage}
\fancyfoot[C]{}

\chapter*{Appendixes}

Additional stuff

\FloatBarrier

\newpage
\fancyhead[L]{References}
\fancyhead[R]{\thepage}
\fancyfoot[C]{}
\addcontentsline{toc}{section}{Appendixes}

\chapter*{REFERENCES}
\addcontentsline{toc}{section}{References}

\setlength{\parindent}{-0.5in}
\setlength{\leftskip}{0.4in}
\setlength{\parskip}{6pt}

\noindent


\end{document}
